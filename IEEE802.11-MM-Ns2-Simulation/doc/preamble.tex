\usepackage{listings}
\usepackage{courier}
\usepackage{color}
\usepackage{float}
\usepackage{pdfpages}
\usepackage{colortbl}
\usepackage{latexsym,bm,amsmath,amssymb}
\usepackage[parfill]{parskip}
\usepackage{array}
\usepackage{multirow}
\usepackage{slashbox}
\usepackage[ pdftex, plainpages = false, pdfpagelabels, 
             pdfpagemode = UseOutlines,
             bookmarks = true,
             bookmarksopen = true,
             bookmarksnumbered = true,
             breaklinks = true,
             linktocpage = false,
             pagebackref,
             colorlinks = true,
             linkcolor = black,
             urlcolor  = black,
             citecolor = black,
             anchorcolor = black,
             hyperindex = true,
             hyperfigures,
             pdfstartview = FitH
             ]{hyperref} 
\usepackage[all]{hypcap}

\lstset{
		basicstyle=\footnotesize\ttfamily, % Standardschrift
		numberstyle=\tiny,          % Stil der Zeilennummern
		numbersep=5pt,              % Abstand der Nummern zum Text
		tabsize=2,                  % Groesse von Tabs
		extendedchars=true,         %
		breaklines=true,            % Zeilen werden Umgebrochen
		keywordstyle=\color{red},
		frame=b,         
		stringstyle=\color{white}\ttfamily, % Farbe der String
		showspaces=false,           % Leerzeichen anzeigen ?
		showtabs=false,             % Tabs anzeigen ?
		xleftmargin=17pt,
		framexleftmargin=17pt,
		framexrightmargin=5pt,
		framexbottommargin=4pt,
		showstringspaces=false      % Leerzeichen in Strings anzeigen ?        
		}

\lstloadlanguages{
        Tcl
		}
	    \usepackage{caption}
		\DeclareCaptionFont{black}{\color{black}}
		\DeclareCaptionFormat{listing}{\colorbox[cmyk]{0,0,0,0}{\parbox{\textwidth}{\hspace{15pt}#1#2#3}}}
		\captionsetup[lstlisting]{format=listing,labelfont=black,textfont=black, singlelinecheck=false, margin=0pt, font={bf,footnotesize}}

\makeatletter
    \renewcommand{\thefigure}{%
        \thesection.\arabic{figure}}
        \@addtoreset{figure}{section}
\makeatother

\makeatletter
    \renewcommand{\thetable}{%
        \thesection.\arabic{table}}
        \@addtoreset{table}{section}
\makeatother

\newcounter{myfootertablecounter}

\newcommand\myfootnotemark{%
%\refstepcounter{footnote}%
\addtocounter{footnote}{1}%
\footnotemark[\thefootnote]%
}%

\newcommand\myfootnotetext[1]{%
\addtocounter{myfootertablecounter}{1}
\footnotetext[\value{myfootertablecounter}]{#1}
}

% from now on, myfootnote has to be used rather than footnote to
% adapt the myfootercounter
\newcommand\myfootnote[1]{%
\addtocounter{myfootertablecounter}{1}
\footnote{#1}
}%
